\chapter{Reflection}\label{ch:reflection}

Over the course of my thesis, I've learned a lot about software, formal methods, mathematics and verification, but also
many things about research, learning and self improvement. Through the course of my project I've also learned many things
that have stood out from my industry experiences, that have often caused me re-evaluate my perspectives on work and research.

\section{Research Is Patient, Industry is Practical}

Often during my industry experience I counter very solvable and very common problems, most of which have a straight forward
and done-before solution. I feel that often when confronted with a problem that's unfamiliar, it's not too long until
a search reveals a solution to excatly what you need, or someone is able to tell you exactly what to do. Of course,
this isn't true of all industry jobs however my experiences leave me with the impression that this is commonplace in many.

During my experiences with research, I have so far found that while finding solutions to problems takes a similar journey
to industry experiences, the solutions found take a level of thought before proper understanding is achieved or a pracitcal
solution can be developed. Answers to problems often involve ideas or techniques that exist only in particular research
that merely reading a paper or asking my supervisor cannot solve. I often find myself sitting down with a pen and paper
going through examples and cases slowly in order to properly understand what I need to know.

While industry problems have previously given me a practical satisfaction and often something I can show or be proud of,
I feel my thesis often leaves me with an understanding of very deep results that show up in weird and interesting places.
While I usally can't explain to my parents or friends what it is I've learned or created, I feel the things I've learned and
the concepts I've come grasp in my own way will serve me for the rest of my life in many facets in and out of research and
industry.

\section{The Things I've Learned from Research}

One of the most important things I've gained from my thesis is the ability to slowly break down and understand high level
concepts. Often when reading very technical papers I struggle to understand things at first sight if they are unfamiliar,
and while at first I found it very difficult to gain anything from this kind of research, over time I've learned that
sitting down and taking the time to walk yourself through the concept goes a long way, no matter how impossible to grasp
a concept seems to be. Of course this understanding doesn't apply to just reading technical papers --- the ability to
be patient when learning difficult concepts goes a long way, and I feel I understand this more than before.

On the practical side, I feel much more comortable reading and writing mathematics and formal descriptions. In addition to
this, I feel I have solidified my understanding of the importance of type systems in software development and
type driven development. I have a heightened appreciation for the need for software correctness, and how using
tools based on formal methods and concepts can facilitate correct software and help write bug-free software.