\chapter{Project Preparations}\label{ch:preperation}

Prior to my thesis year, I took the subjects COMP2111, COMP3141, COMP3161, COMP3891 and COMP4161,
all of which have provided me with knowledge I require to progress with my thesis project.

COMP2111 (System Modelling and Design) gave me adequate knowledge of the basis of verification 
--- practice with developing
mathematical models to describe the semantics of programs and their states by means of 
axiomatic semantics, Hoare logic and creating invariants for simple programs. I also 
learned about methods of data refinement, whereby abstract representations are refined to low
level implementations via a specification.

In COMP3141 (Software System Design and Implementation) I was introduced to functional 
programming through Haskell, which is of great
use to my progress with my project as Cogent is a functional programming language implemented in Haskell.
As well as this I learned the basics of more advanced formal methods, for example the Curry-Howard
Correspondence and working with types to ensure correctness of programs.

In COMP3161 (Concepts of Programming Languages), I was able to learn more Haskell and the
basics of programming language theory and type theory by implementing the small programming 
language minhs in Haskell. As my project is heavily involved with programming language development,
specifically in Cogent's type system, this subject is very relevant to my project and thus the knowledge
I learned from it will be essential to my success.

In COMP3891 (Extended Operating Systems), I learned background knowledge about systems development,
how lower level systems work and various implementation strategies for these systems. As Cogent
is a domain specific language for systems development, this knowledge will benefit me throughout
the project as it has throughout my research when understanding the context of Cogent.

In COMP4161 I learned more advanced methods in formal verification, and gained experience with using
the theorem prover Isabelle. In addition to this, I gained greater experience in understanding the 
terminology and concepts behind various formal methods areas, such as reading natural deduction,
understanding invariants and preservation of correctness, proofs about termination and totality
of programs, and correctness with respect to specifications.

The skills I have gained in my university subjects so far have provided me with concrete knowledge on
how to progress with my thesis project and understand the underlying theory and core concepts behind
it.